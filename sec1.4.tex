\section{Hyperplanes and linear forms}\label{sec:1.4}
A linear subspace $W$ of a vector space $V$ is called a hyperplane if
$\mathrm{codim}\,W=1$. Equivalently, this means that $W$ is a proper
subspace of $V$ contained in no strictly proper subspace. A hyperplane
can always be defined by one linear equation. For if $V/W$ has
dimension $1$, there is a bijection $V/W\to K$. By composition we get
a \emph{linear form}
\begin{displaymath}
  L:V\to V/W\to K
\end{displaymath}
such that $W$ is the inverse image of $0$ in $K$. Thus we have
\begin{equation}
  \label{eq:1.4.1}
  L(x+y)=L(x)+L(y),\ x,y\in V;\quad L(ax)=aL(x),\ a\in K,x\in
  V;
\end{equation}
and $L(x)=0$ if and only if $x\in W$. Conversely, assume that we have
a linear form $L$, that is, a map $V\to K$ satisfying
\eqref{eq:1.4.1}, and that $L$ does not vanish identically. Then
$W=\{x;L(x)=0\}$ is a hyperplane, for the map $V/W\to K$ induced by
$L$ is a bijection.\par
Now consider a linear subspace $W$ of finite codimension
$n$. Composing the quotient map $V\to V/W$ with a bijection $V/W\to
K^n$ we then get a linear map $L:V\to K^n$ such that $L$ is surjective
and $Lx=0$ is equivalent to $x\in W$. Writing $L=(L_1,\dots,L_n)$,
where $L_j$ are linear forms, we conclude that $W$ can be defined by
$n$ linear equations, that is, $W$ is the intersection of $n$
hyperplanes. On the other hand, the intersection of $k$ hyperplanes
has codimension $\leq k$ in view of \eqref{eq:1.2.8}. Thus we have
\begin{thm}
  \label{thm:1.4.1}
  A linear subspace $W$ of $V$ of codimension $n$ can always be
  obtained as the intersection of $n$ but not fewer than $n$ hyperplanes.
\end{thm}
We shall now discuss an analogue of Theorem \ref{thm:1.4.1} for spaces
of arbitrary codimension.
\begin{thm}
  \label{thm:1.4.2}
  Every linear subspace of $V$ is the intersection of hyperplanes.
\end{thm}
An equivalent statement is the following:
\addtocounter{dfn}{-1}
\renewcommand{\thedfn}{\arabic{chapter}.\arabic{section}.\arabic{dfn}$'$}
\begin{thm}
  \label{thm:1.4.3}
  If $V_1$ is a subspace of $V$ and $x$ an element in $V\setminus V_1$, then
  one can find a hyperplane $V_2$ in $V$ such that $V_1\subset V_2$
  but $x\notin V_2$.
\end{thm}
\renewcommand{\thedfn}{\arabic{chapter}.\arabic{section}.\arabic{dfn}}
\begin{prf}
  \begin{enumerate}[label=\alph*),labelindent=\parindent,itemindent=!,leftmargin=0pt]
  \item\label{item:1.3}  We prove first that if $\codim V_1>1$ then one can choose
    $V_2$ containing $V_1$ strictly but not containing $x$. To do so
    we note that $V/V_1$ has dimension $>1$ and that the residue class
    $\dot{x}$ of $x$ there is not $0$. We can therefore find another
    element, say the residue class $\dot{y}$ of some $y\in V$, which
    is linearly independent of $\dot{x}$. Now set
    \begin{displaymath}
      V_2=\left\{ z+ty;z\in V_1,t\in K \right\}.
    \end{displaymath}
    This is a linear subspace of $V$ which contains $V_1$ strictly
    since $y\in V_2\setminus V_1$. On the other hand, $x\notin V_2$
    for the image of $V_2$ in $V/V_1$ is generated by $\dot{y}$ so it
    does not contain $\dot{x}$.
  \item\label{item:1.4} Now consider the set $\mathcal{F}$ of all linear subspaces
    $V_2$ of $V$ containing $V_1$ but not $x$. The union of a
    completely ordered subset of $\mathcal{F}$ is obviously an element
    of $\mathcal{F}$. Accroding to Zorn's lemma it follows that there
    exists at least one maximal element $V_2\in\mathcal{F}$. Since
    $x\notin V_2$ we have $\codim V_2\geq 1$, and by \ref{item:1.3} we have that
    $V_2$ would not be maximal if $\codim V_2>1$. Hence $\codim
    V_2=1$, which proves the theorem.
  \end{enumerate}
\end{prf}
With $V_1=\{0\}$ we conclude in particular that hyperplanes exist and
that for every $x\in V\setminus 0$ there is some linear form $L$ on
$V$ with $L(x)\neq0$. Another useful consequence is given by
\begin{cor}\label{cor:1.4.3}
  If $W_1\subset V$ is a linear subspace of finite dimension, then
  there exists a linear subspace $W_2\subset V$ with $\codim W_2=\dim
  W_1$ such that $W_1\cap W_2=\{0\}$. Thus $V$ is the direct sum
  $W_1\oplus W_2$.
\end{cor}
\begin{prf}
The assertion is obvious if $W_1=\{0\}$. If $0\neq x\in W_1$ we can
find a hyperplane $H_1$ with $x\notin H_1$, thus $\dim(W_1\cap
H_1)=\dim W_1-1$ by \eqref{eq:1.2.9} since $H_1+W_1=V$. Repeating the
argument we obtain hyperplanes $H_k,\ k=1,\ldots,d,\ d=\dim W_1$, such
that
\begin{displaymath}
  \dim(W_1\cap H_1\cap\cdots\cap H_k)=d-k,\qquad k=1,\ldots,d.
\end{displaymath}
Thus $W_1\cap W_2=\{0\}$ if $W_2=H_1\cap\cdots\cap H_d$, and
$d\leq\codim W_2\leq d$ by \eqref{eq:1.2.9} and \eqref{eq:1.2.8}.
\end{prf}
\begin{rem}
  Using Zorn's lemma it is also easy to prove directly that Corollary
  \ref{cor:1.4.3} holds for an arbitrary linear subspace $W_1$. We
  leave this as an exercise for the reader.
\end{rem}
As an application we prove a supplement to Theorem \ref{thm:1.3.1}.
\begin{thm}
  \label{thm:1.4.4}
  Let $T:V_1\to V_2$ be a linear map of index $0$. Then one can find a
  linear map $S:V_1\to V_2$ of finite rank such that $T+S$ is a bijection.
\end{thm}
\begin{prf}
  Using Theorem \ref{cor:1.4.3} we choose a linear subspace $W_1$ of
  $V_1$ such that $W_1\cap \Ker T=\{0\}$ and $\codim W_1=\dim\Ker
  T$. The range of $T$ is then equal to the range of the restriction
  of $T$ to $W_1$, which is injective. Now choose a linear subspace
  $W_2\subset V_2$ such that $W_2\cap\im T=\{0\}$ and $\dim
  W_2=\dim\coker T$. We just have to take $W_2$ spanned by vectors
  whose inages in $\coker T$ form a basis there. By hypothesis
  \begin{displaymath}
    \dim\Ker T=\dim\coker T=\dim W_2;
  \end{displaymath}
hence we can define $S$ so that $S$ vanishes on $W_1$ and the
restriction to $\Ker T$ is a bijection with range $W_2$. Then the
image of $T+S$ contains $\im T$ and $W_2$ so it is equal to $V_2$. The
construction also shows that $T+S$ is injective. This proves the theorem.
\end{prf}
The origin in $V$ belongs to every linear subspace. Sometimes it is
convenient to remove this special role of the origin by introducting
the following concept:
\begin{dfn}
  \label{def:1.4.5}
  A subset $W$ of the vector space $V$ is called an \emph{affine}
  subspace (hyperplane) if $W$ can be transformed to a linear subspace
  (hyperplane) by a translation.
\end{dfn}
The definition means that $\{x-y;x\in W\}$ is a linear subspace
(hyperplane) for every fixed $y\in W$, and that this set is
independent of $y$. Conversely, if $\{x-y;x\in W\}$ is a linear
subspace (hyperplane) for some fixed $y$, then $W$ is an affine
subspace (hyperplane) through $y$. With this terminology
Theorem \ref{thm:1.4.3} extends immediately to affine spaces.

Every linear form $L$ on $K^n$ is obviously of the form
\begin{displaymath}
  K^n\ni (a_1,\ldots,a_n)\mapsto \sum_1^nc_ja_j
\end{displaymath}
for some $c_j\in K$. However, in an infinite dimensional space there
may be so many linear forms that they are hard to describe. This is
one reason for the study of vector spaces with topology where one only
has to consider \emph{continuous} linear forms. As a preparation for
that we shall now give an extension of Theorem \ref{thm:1.4.3} with
$K=\mathbf{R}$ and the point replaced by a larger set.
\begin{dfn}
  A subset $A$ of the vector space $V$ over $\mathbf{R}$ is called
  \emph{convex} if for arbitrary $x,y\in V$ the set $\{t;t\in
  \mathbf{R},x+ty\in A\}$ is an interval (open, closed or half open;
  finite or infinite). We say that $A$ is convex and \emph{linearly
    open} if the interval is always open.
\end{dfn}
\begin{thm}
  (Geometric form of the Hahn-Banach theorem.) Let $A$ be a convex
  linearly open set in the vector space $V$ over $\mathbf{R}$, and let
  $V_1$ be a linear subspace which does not intersect $A$. Then there
  exists a hyperplane $V_2$ such that $V_1\subset V_2$ and $V_2\cap
  A=\emptyset$.
\end{thm}
\begin{prf}
  This is analogous to the proof of Theorem \ref{thm:1.4.3} although
  the $2$-dimensional case is not equally trivial now. We discuss it
  first.
  \begin{enumerate}[label=\alph*),labelindent=\parindent,itemindent=!,leftmargin=0pt]
  \item $\dim V=2$. Then $V_1=\{0\}$ or else there is nothing to
    prove. From the convexity of $A$ it follows that if a half ray
    $\{tx; t\geq0\}$ through $0$ intersects $A$, then the opposite
    half ray $\{tx; t\leq0\}$ does not. Identifying half rays with
    points on the unit circle by means of a basis in $V$, we denote by
    $O_{+}$ and $O_{-}$ the set of half rays intersecting $A$ and
    their opposites. There are open sets since $A$ is linearly open,
    and we have just seen that they are disjoint. Since the unit
    circle is connected, it follows that it cannot be the union of
    $O_{+}$ and $O_{-}$, so we can find a half ray which is neither in
    $O_{+}$ nor in $O_{-}$. The corresponding line has no point in
    common with $A$ then.
  \item Assuming that $\codim V_1>1$ we now prove that there exists a
    strictly larger subspace $V_2$ which does not intersect $A$. Thus
    form $V'=V/V'$ and let $A'$ be the image of $A$ in $V'$. Then $A'$
    is convex and linearly open. For if $\xi_j\in A'$ for $j=1,2$, we
    can find $x_j\in A$ with residue class $\xi_j$ for $j=1,2$. Thus
    we have $tx_1+(1-t)x_3\in A$ for all $t$ in an open interval $I$
    containing $[0,1]$, hence $t\xi_1+(1-t)\xi_2\in A',\ t\in I$,
    which does not intersect $A'$. In fact, we can choose $W'$ in any
    two dimensional subspace of $V'$ (which by assumption has
    dimension $>1$). But then the inverse image of $W'$ in $V$ by the
    quotient map $V\to V'$ had the required properties.
  \item We can now apply Zorn's lemma precisely as in part
    \ref{item:1.4} of the proof of Theorem \ref{thm:1.4.3}. The
    repetition is left as an exercise for the reader.
  \end{enumerate}
\end{prf}

%%% Local Variables: 
%%% mode: latex
%%% TeX-master: "alfa"
%%% End: 
