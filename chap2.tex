\chapter{Topological vector space}
\def\O{\mathcal{O}}
\def\V{\mathcal{V}}
\def\N{\mathcal{N}}
\def\T{\mathcal{T}}
\def\F{\mathcal{F}}
 \newcommand{\eqdef}{%
\ensuremath{\mathrel{\stackrel{\mathrm{def}}{=}}}}
\section{Topological and metric spaces}
Let us first recall that a set $E$ is called a topological space if
there is given a family $\O$ of subsets, which are said to be
\emph{open}, such that 
\begin{enumerate}[label=\alph*),itemindent=!,leftmargin=0pt]
\item\label{item:2.1} $E\in\O$,
\item\label{item:2.2} the union of the sets in any subfamily of $\O$ is always a set in $\O$,
\item\label{item:2.3} the intersection of finitely many sets in $\O$ is also in $\O$.
\end{enumerate}
The space $E$ is called a Hausdorff space if in addition
\begin{enumerate}[resume*]
\item\label{item:2.4} for two arbitrary different points $x,y\in E$ one can always
  find two disjoint sets in $\O$ containing $x$ and $y$ respectively.
\end{enumerate}\par
The complements in $E$ of the open sets are said to be
\emph{closed}. According to \ref{item:2.2} and \ref{item:2.3}
\emph{arbitrary intersections and finite unions of closed sets are
  closed.} For every set $M\subset E$ the intersection $\overline{M}$
of all closed sets containing $M$ is therefore closed, hence the
smallest closed set containing $M$.

A subset $N$ of $E$ is called a neighborhood of $x\in E$ (and $x$ is
said to be an interior point of $N$) if $x\in O\subset N$ for some
$O\in\O$. The neighborhoods of $x$ have the following properties:
\begin{enumerate}[label=\roman*),itemindent=!]
\item\label{item:2.5} $E$ is a neighborhood of $x$, and $x$ belongs to
  every neighborhood of $x$,
\item\label{item:2.6} the intersection of a finite number of
  neighborhoods of $x$ is a neighborhood of $x$,
\item\label{item:2.7} if $N$ is a neighborhood of $x$ and $N'\supset
  N$, then $N'$ is a neighborhood of $x$,
\item\label{item:2.8} if $N$ is a neighborhood of $x$ then there is
  another neighborhood $N'$ of $x$ such that $N$ is a neighborhood of
  every $y\in N'$.
\end{enumerate}
From the neighborhoods one can reconstruct the open sets: If $O$ is a
subset of $E$ then $O\in\O$ if (and only if) $O$ is a neighborhood of
all of its points. In fact, for every $x\in O$ we can then find
$O_x\in\O$ with $x\in O_x\subset O$. Hence $\cup_{x\in O}O_x=O$ is in
$\O$ by condition \ref{item:2.2}.

Suppose now instead that for every $x\in E$ we are given a family
$\V_x$ of subset of $E$ called neighborhoods of $x$, so that
\ref{item:2.5} and \ref{item:2.6} are fulfilled. We can then define a
topology $\T$ in $E$ by declaring that a set $O\subset E$ is open if
for every $x\in O$ we have $N\subset O$ for some $N\in\V_x$. Then
\ref{item:2.1} follows from \ref{item:2.5}, \ref{item:2.2} is trivial,
and \ref{item:2.3} follows from \ref{item:2.6} or even the weaker
version
\begin{enumerate}[label=\roman*)$'$,start=2]
\item\label{item:2.9} the intersection of a finite number of
  neighborhoods of $x$ contains a neighborhood of $x$.
\end{enumerate}
If $\N$ is a neighborhood of $x$ in the topology $\T$ thus defined,
then we can find an open set $O$ with $x\in O\subset\N$, hence some
$N'\in\V_x$ with $x\in N'\subset O\subset\N$, so $\N$ contains one of
our original ``neighborhoods''. Now suppose that \ref{item:2.8} is
fulfilled, and let $N\in\V_x$. Denote by $O$ the set of all $y\in N$
such that $N\in\V_y$. By \ref{item:2.8} we can then find $N_y'\in\V_y$
such that $N\in\V_z$ for every $z\in N'_y$, thus $N'_y\subset O$. This
means that $O$ is open in the topology $\T$ and since $x\in O\subset
N$ we see that $N$ is also a neighborhood of $x$ in the topology
$\T$. The neighborhoods of $x$ in the topology $\T$ are therefore
precisely the sets containing some set belonging to $\V_x$. If
\ref{item:2.7} is fulfilled the neighborhoods are precisely those we
were given from the beginning.

To define a topology we can therefore either give a family of open
sets satisfying \ref{item:2.1},~\ref{item:2.2},~\ref{item:2.3} or give
the families $\V_x$ of neighborhoods satisfying at least
\ref{item:2.5},~\ref{item:2.9},~\ref{item:2.8}. If condition
\ref{item:2.7} is not required one calls the sets $\V_x$ a
\emph{neighborhood basis} or a \emph{fundamental system of
  neighborhoods}. The neighborhoods of $x$ in the topology defined are
precisely the sets which contain some set belonging to $\V_x$.

Any subset $E_1$ of $E$ is itself a topological space with the
restriction of the topology in $E:$ The open sets in $E_1$ are by
definition the sets $O\cap E_1$ with $O\in\O$.

A map $f:E_1\to E_2$ between topological spaces is called
\emph{continuous} if for every open set $O\subset E_2$ the
\emph{inverse image}
\begin{displaymath}
  f^{-1}(O)\eqdef\{x\in E_1; f(x)\in O\}
\end{displaymath}
is an open set. Equivalently this means that for every neighborhood
$N_2$ of $f(x)$ one can find a neighborhood $N_1$ of $x$ with
$f(N_1)\subset N_2$.

A sequence $x_j\in E$ is said to converge to $x\in E$ if for any
neighborhood $N$ of $x$ all but a finite number of the points $x_j$
belong to $N$. If the Hausdorff separation axiom \ref{item:2.4} above
is satisfied, then this cannot happen for more than one value of
$x$. A closed set $F\subset E$ contains all limits of sequences
contained in $F$, for $E\setminus F$ is a neighborhood of all of it
points and contains no element of such a sequence.

A topology can sometimes be defined by means of a \emph{metric}, that
is, a function of $d$ on $E\times E$ with values in the non-negative
reals such that
\begin{enumerate}[label=\alph*),itemindent=!,leftmargin=0pt]
\item\label{item:2.10} $d(x,z)\leq d(x,y)+d(y,z)$, (the triangle
  inequality),
\item\label{item:2.11} $d(x,y)=d(y,x)$,
\item\label{item:2.12} $d(x,y)=0\Longleftrightarrow x=y$,
\end{enumerate}
for all $x,y,z\in E$. The metric gives rise to a system of
neighborhoods of $x$,
\begin{displaymath}
  N_{x,\varepsilon} = \{y\in E; d(x,y)<\varepsilon\},\qquad
  0<\varepsilon\leq\infty,\ x\in E,
\end{displaymath}
satisfying \ref{item:2.5},~\ref{item:2.6},~\ref{item:2.8} above, so
the metric defines a topology. It follows from \ref{item:2.10} that
$d(x,y)$ is a continuous function of $x$ (or $y$) in this topology. It
is clear that the topology satisfies the Hausdorff separation
condition \ref{item:2.4} above. (Note that all topologies cannot be
defined by a metric, and that different metrics may define the same
topology.)

A metric space is called \emph{complete} if for every \emph{Cauchy
  sequence}, that is, every sequence $x_n\in E$ with $d(x_n,x_m)\to 0$
as $n,m\to\infty$, there is an element $x\in E$ such that $x_n\to x$,
that is, $d(x_n,x)\to 0$. The real numbers $\mathbf{R}$ and the
complex numbers $\mathbf{C}$ with the usual distance $d(x,y)=|x-y|$
for $x,y\in \mathbf{C}$ are examples of complete metric spaces.

Some of the most important theorems in functional analysis follow from
a fundamental classical theorem of Baire concerning complete metric
spaces:
\begin{thm}\label{thm:2.1.1}
  (Baire) Let $E$ be a complete metric space, and let $F_n,\ 
  n=1,2,\dots,$ be closed subsets containing no interior points. Then
  the union $\cup_1^{\infty}F_n$ has no interior point either.
\end{thm}
\begin{prf}
  For any $x\in E$ and $\varepsilon>0$ we want to show that
  \begin{displaymath}
    N_{x,\varepsilon} = \left\{ y\in E;d(x,y)\leq\varepsilon \right\}
  \end{displaymath}
is not contained in $\cup F_n$. To do so we want to choose sequences
$x_j\in E$ and $\varepsilon_j>0$ so that
\begin{displaymath}
  N_{x,\varepsilon}\supset N_{x_1,\varepsilon_1}\supset
  N_{x_2,\varepsilon_2}\supset\cdots,\qquad F_j\cap
  N_{x_j,\varepsilon_j}=\emptyset \quad \text{for every } j.
\end{displaymath}
When $j<k$ this implies that $x_k\in N_{x_j,\varepsilon_j}$, hence
$d(x_j,x_k)<\varepsilon_j$, so $x_k$ is a Cauchy sequence if
$\varepsilon_j\to 0$. The limit $y$ belongs to $N_{x_j,\varepsilon_j}$ for
every $j$ since $N_{x_j,\varepsilon_j}$ is closed. Hence $y\notin\cup
F_j$ so the theorem will follow if we show that the sequences
$x_j,\varepsilon_j$ can be constructed. Suppose that
$x_1,\varepsilon_1,\dots,x_{j-1},\varepsilon_{j-1}$ have already been
chosen. By hypothesis the ball $N_{x_{j-1},\varepsilon_{j-1}/2}$ contains
some point $x_j$ which is not in $F_j$. Since $F_j$ is closed we can
choose $\varepsilon_j$ so that $F_j\cap N_{x_j,\varepsilon_j}=\emptyset$. If
$\varepsilon_j<\varepsilon_{j-1}/2$, we have $N_{x_j,\varepsilon_j}\subset
N_{x_{j-1},\varepsilon_{j-1}}$ by the triangle inequality, and
$\varepsilon_j<2^{1-j}\varepsilon_1\to 0$. This proves the theorem.
\end{prf}
\begin{exa}
  We give an example with $E=\mathbf{R}$. Let $f$ be a function from
  $\mathbf{R}$ to $\mathbf{R}$ which is differentiable at every
  point. The example $f(x)=x^2\sin(x^{-2}),\ x\neq0;\ f(0)=0$; shows
  that $\varlimsup_{y\to x}|f'(y)|$ need not be finite for every
  $x$. However, we shall prove that
  \begin{displaymath}
    E=\left\{ x\in\mathbf{R};\varlimsup_{y\to x}|f'(y)|<\infty \right\}
  \end{displaymath}
is an open dense set (that is, the complement is closed and has no
interior point). That $E$ is open is obvious. Let $I$ be a closed
interval in $\mathbf{R}$, not reduced to a point, and set
\begin{displaymath}
  F_n=\left\{ x\in I;|f(x)-f(y)|\leq n|x-y|,y\in I \right\}\subset
  \left\{ x\in I;|f'(x)|\leq n \right\}.
\end{displaymath}
Since $f$ is continuous it is clear that $F_n$ is closed, and $\cup
F_n=I$ since $f'(x)$ exists for every $x$. Hence $F_n$ has an interior
point for some $n$, so $I\cap E$ is not empty, which means that $E$ is
dense in $\mathbf{R}$.
\end{exa}
\begin{dfn}
  A subset $A$ of complete metric space is said to be of the
  \emph{first category} if there exist countably many closed sets
  $F_1,F_2,\dots$ in $E$ without interior points such that
  $A\subset\cup_1^{\infty}F_j$. All other sets are said to be of the
  \emph{second category}.
\end{dfn}
The definition implies that any countable union of sets of the first
category is of the first category, and so is any subset of a set of the
first category. By Theorem \ref{thm:2.1.1} a set of the first category
has no interior point, so the complement is dense. One is therefore
justified in thinking of sets of the first category as quite small
although they may of course be dense (such as the set of rational
numbers $\subset\mathbf{R}$).

We shall finally recall the main facts concerning compact spaces. A
topological space $E$ is called \emph{compact} if it is a Hausdorff
space and the Borel-Lebesgue lemma is valid, that is, if for every
family of open subsets $O_{\alpha},\alpha\in A$, with
$\cup_AO_{\alpha}=E$ it is possible to find a finite subfamily
$O_{\alpha_1},\dots,O_{\alpha_n}$ with union equal to $E$. Here $A$
may be any set of indices. An equivalent property is that for closed
subsets $F_{\alpha}$ of $E$ with $\cap_AF_{\alpha}=\emptyset$ a finite
number of the sets already have an empty intersection. This follows by
considering the complements. Negation gives a third equivalent and
useful statement:

If $F_{\alpha}$ are closed subsets of the compact set $E$ and if any
finite number of the sets $F_{\alpha}$ have a non-empty intersection,
then all have a non-empty intersection.

It is obvious that \emph{a closed subset of a compact space is
  compact}. The converse is contained in the following:
\begin{thm}\label{thm:2.1.3}
  (i) Every compact subset of a Hausdorff space is closed. (ii) If
  $f:E\to F$ is continuous, $E$ is compact and $F$ is Hausdorff, then
  $F(E)$ is compact. (iii) If in addition $f$ is injective, then $f$
  is a homeomorphism $E\to f(E)$, that is, the inverse is also
  continuous. (iv) If $E$ is compact, then every $x\in E$ has a
  fundamental system of compact neighborhoods.
\end{thm}
\begin{prf}
  (i) Let $K$ be a compact subset of the Hausdorff space $E$ and let
  $x\in E\setminus K$. For every $y\in K$ we can choose disjoint open
  sets $O'_y\ni x$ and $O''_y\ni y$. Since $K$ is compact, we can find
  $y_1,\dots,y_k$ so that $K\subset O''_{y_1}\cup\cdots\cup O''_{y_k}
  $. Then $O'_{y_1}\cap\cdots\cap O'_{y_k}$ is an open set containing
  $x$ which does not intersect $K$. Hence $E\setminus K$ is an open
  set. (ii) If $O_{\alpha}$ are open subsets of $F$ with
  $f(E)\subset\cup O_{\alpha}$, then $E\subset\cup
  f^{-1}(O_{\alpha})$, so $E\subset\cup_1^kf^{-1}(O_{\alpha_j})$ for
  suitable $\alpha_1,\dots,\alpha_k$. This implies that
  $f(E)\subset\cup O_{\alpha_j}$ so $f(E)$ is compact set since it is
  clearly Hausdorff. (iii) From (ii) it follows that $f$ maps closed
  subsets of $E$ to closed subsets of $f(E)$, hence open subsets to
  open subsets, which means that $f^{-1}$ is continuous. (iv) If $O$
  is an open neighborhood of $x$ then the proof of (i) applied to
  $K=E\setminus O$ shows that there are disjoint open sets $O'\ni x$
  and $O''\supset K$; then $O'\subset N=E\setminus O''\subset O$, and
  since $N$ is closed, hence compact, the statement is proved.
\end{prf}

Let $A$ be an arbitrary set and let $E_{\alpha}$ be a compact set for
every $\alpha\in A$. The infinite product
\begin{displaymath}
  E=\prod_{\alpha\in A}E_{\alpha}
\end{displaymath}
is defined as the set of all systems $\left\{ x_{\alpha}
\right\}_{\alpha\in A}$ with $x_{\alpha}\in E_{\alpha}$. Let
$p_{\alpha}$ be the projection $E\to E_{\alpha}$ on the $\alpha$th
component. A topology in $E$ is defined by taking as a basis for open
sets the finite intersections of sets of the form
$p_{\alpha}^{-1}O_{\alpha}$ where $O_{\alpha}$ is open in
$E_{\alpha}$; note that this makes the projections $p_{\alpha}$
continuous.
\begin{thm}
  (Tychonov) The inifinite product $E=\prod_{\alpha\in A}E_{\alpha}$
  is compact if each $E_{\alpha}$ is compact.
\end{thm}
\begin{prf}
  Let $\F$ be a family of closed subsets of $E$ such that finitely
  many members of $\F$ never have an empty intersection. Using Zorn's
  lemma we can extend $\F$ to a maximal family $\F'$ having the same
  property. In particular, finite intersections of sets in $\F'$ are
  themselves in $\F'$. Since $E_{\alpha}$ is compact we can find
  $x_{\alpha}\in E_{\alpha}$ so that
  $x_{\alpha}\in\overline{p_{\alpha}F}$ for every $F\in\F'$, where the
  bar denotes closure. If $U_{\alpha}$ is a compact neighborhood of
  $x_{\alpha}$, then it follows that $p_{\alpha}^{-1}U_{\alpha}\in\F'$
  since $\F'$ is maximal. Hence $x_{\alpha}$ is uniquely
  determined. Choosing all $U_{\alpha}$ except finitely many equal to
  $E_{\alpha}$, we obtain
  \begin{displaymath}
    \prod U_{\alpha}=\bigcap
    p_{\alpha}^{-1}U_{\alpha}\in\F',\quad\text{hence } \left( \prod
      U_{\alpha} \right)\cap F\neq\emptyset \quad \text{if } F\in\F'
  \end{displaymath}
  Thus an arbitrary neighborhood of $x=\left\{ x_{\alpha}
  \right\}_{\alpha\in A}$ intersects $F$, for every $F\in\F'$, and
  since $F$ is closed this implies that $x\in F$. The proof is complete.
\end{prf}

%%% Local Variables: 
%%% mode: LaTeX
%%% TeX-master: "alfa"
%%% End: 

\section{Vector space topologies}
A vector space $V$ which is also a topological space is call a
\emph{topological vector space} if the vector operations
\begin{equation}
  \label{eq:2.2.1}
  V\times V\ni (x,y)\mapsto x+y\in V; \quad K\times V\ni (a,x)\mapsto
  ax\in V
\end{equation}
are continuous. Here and in what follows $K$ denotes the field
$\mathbf{R}$ of real numbers or the field $\mathbf{C}$ of complex
numbers with the usual metric topologies. In particular,
\eqref{eq:2.2.1} implies that for fixed $y$ and fixed $a\in K, a\neq
0$, the maps
\begin{displaymath}
  x\mapsto x+y \quad (\text{translation})\qquad\text{and } x\mapsto ax \quad(\text{dilation})
\end{displaymath}
are homeomorphisms. Thus the open sets are invariant under
translations, so the topology is completely determined by the
neighborhoods of $0$. Since
\begin{displaymath}
  x+x_0+y+y_0=x+y+x_0+y_0,\qquad(a+a_0)(x+x_0)=ax+ax_0+a_0x+a_0x_0,
\end{displaymath}
the continuity of the operations \eqref{eq:2.2.1} reduces to the
following conditions on the neighborhoods of $0$:
\begin{equation*}
  \label{eq:2.2.1m}
  \begin{aligned}
    (x,y) &\mapsto x+y & {\ }&\text{is continuous at }(0,0);\\
    (a,x) &\mapsto ax  & {\ }&\text{is continuous at }(0,0);\\
    a &\mapsto ax_0 & {\ }&\text{is continuous at }0\text{ for every }x_0;\\
    x &\mapsto a_0x & {\ }&\text{is continuous at }0\text{ for every }a_0.
  \end{aligned}\tag*{$(\ref{eq:2.2.1})'$}
\end{equation*}
Explicitly the first three conditions mean that for every neighborhood
$N$ of $0$ in $V$ there shall exist a neighborhood $N_1$ of $0$ and
$\varepsilon>0$ such that
\begin{equation}
  \label{eq:2.2.2}
  \begin{gathered}
    N_1+N_1=\left\{ x+y;\ x,y\in N_1 \right\}\subset N;\quad ax\in
    N\text{ if }|a|\leq\varepsilon\text{ and }x\in N_1;\\
    \text{for every }x\in V\ \exists\varepsilon_x>0\text{ such that }ax\in
    N \quad\text{if }|a|<\varepsilon_x.
  \end{gathered}
\end{equation}
The last condition in \ref{eq:2.2.1m} is a consequence of
\eqref{eq:2.2.2}, for the second part of \eqref{eq:2.2.2} gives
$a_0x\in N$ if $2^kx\in N_1$ for some integer $k>0$ with
$2^{-k}|a_0|<\varepsilon$, and repeated use of the first part of
\eqref{eq:2.2.2} shows that there is a neighborhood $N_2$ of $0$ such
that $2^kN_2\subset N_1$, thus $a_0x\in N$ if $x\in N_2$.

To use the second part of \eqref{eq:2.2.2} we form
\begin{displaymath}
  N_2=\bigcup_{|a|\leq\varepsilon}aN_1\subset N.
\end{displaymath}
This is also a neighborhood of $0$, and we have $aN_2\subset N_2$ if
$|a|\leq1$.
\begin{dfn}
  A set $M$ in a vector space $V$ over $K$ is called \emph{balanced}
  if $ax\in M$ for all $x\in M$ and $a\in K$ with $|a|\leq1$. It is
  called \emph{absorbing} if for every $x\in V$ we have $ax\in M$ when
  $|a|$ is sufficiently small.
\end{dfn}
Every neighborhood of $0$ is absorbing, and we have found that there
is a fundamental system of balanced neighborhoods of $0$. Conversely,
for any system of balanced absorbing sets satisfying the first
condition in \eqref{eq:2.2.2} the finite intersections can be taken as
a fundamental system of neighborhoods of $0$ for a vector space
topology in $V$.

So far we have not required the Hausdorff separation axiom to be
valid. This axiom is equivalent to
\begin{equation}
  \label{eq:2.2.3}
  \text{the intersection of all neighborhoods of }0\text{ is equal to }\left\{ 0 \right\}.
\end{equation}
In fact, \eqref{eq:2.2.3} is obviously a consequence of the Hausdorff
axiom. On the other hand, if \eqref{eq:2.2.3} is valid we can for
every $x\neq0$ find a balanced neighborhood $N$ of $0$ such that
$x\notin N+N$. The neighborhoods $N$ and $\left\{ x \right\}+N$ of $0$
and $x$ are then disjoint, for if $x+y=z;\ y,z\in N$; then $x=z+(-y)\in
N+N$.

It follows from \eqref{eq:2.2.2} that in any topological vector space
$V$ the intersection $V_0$ of all neighborhoods of $0$ is a linear
subspace. If $O$ is an open set in $V$ and if $x\in O, y\in V_0$, it
follows that $x+y\in O$. The open sets in $V$ are therefore unions of
residue classes modulo $V_0$. We obtain a vector space topology in
$V/V_0$ if we take as open sets there the maps of the open sets in $V$
into $V/V_0$; conversely, the open sets in $V$ are the inverse images
of the open sets in $V/V_0$. The definition of $V_0$ and
\eqref{eq:2.2.3} imply that $V/V_0$ satisfies the Hausdorff axiom.

If $W$ is any linear subspace of a topological vector space $V$, we
can also topologize $V/W$ by taking as open sets the images of the
open sets in $V$. It is clear that $V/W$ is a Hausdorff space if and
only if $0$ is closed there, that is, $W$ is a \emph{closed} linear
subspace of $V$.
\begin{thm}\label{thm:2.2.2}
  If $V$ is a finite dimensional Hausdorff topological vector space
  and if $T:K^n\to V$ is a linear bijection, then $T$ is a
  homeomorphism if $K^n$ is given the product topology.
\end{thm}
Thus there is only one Haudorff vector space topology possible in the
finite dimensional case.
\begin{prf}
  Since for some $x_1,\dots,x_n\in V$
  \begin{displaymath}
    Ta=\sum_1^na_jx_j,\quad a=\left( a_1,\dots,a_n \right)\in K^n,
  \end{displaymath}
  it follows from (\ref{eq:2.2.1}) that $T$ is continuous. To prove
  that $T^{-1}$ is continuous we have to show that $T$ maps open sets
  to open sets. This follows if we prove that $TI$ is a neighborhood
  of $0$ when
  \begin{displaymath}
    I=\left\{ a\in K^n;|a_j|<1,j=1,\dots,n \right\}.
  \end{displaymath}
  If $\partial I$ is the boundary of $I$ then
  \begin{displaymath}
    I=\left\{ a\in K^n;wa\notin\partial I \text{ when }w\in K\text{
        and }|w|\leq1 \right\}.
  \end{displaymath}
  Thus
  \begin{displaymath}
    T(I)=\left\{ x\in V;wx\notin T(\partial I)\text{ when }w\in
      K\text{ and }|w|\leq1 \right\}.
  \end{displaymath}
  Since $T(\partial I)$ is a compact set which does not contain $0$
  (Theorem \ref{thm:2.1.3}), it follows, again by Theorem
  \ref{thm:2.1.3}, that there is a neighborhood $N$ of $0$ with $N\cap
  T(\partial I)=\emptyset$. If we take $N$ balanced it follows that
  $N\subset T(I)$.
\end{prf}
\begin{cor}\label{cor:2.2.3}
  Let $W$ be a linear subspace of finite codimension in a topological
  vector space $V$. Then all linear forms vanishing in $W$ are
  continuous if and only if $W$ is closed.
\end{cor}
\begin{prf}
  It follows from Theorem \ref{thm:1.4.2} that $w$ is the intersection
  of the zero sets of linear forms vanishing on $W$, so $W$ is closed
  if these are all continuous. On the other hand, if $W$ is closed
  then $V/W$ is a finite dimensional Hausdorff space. A linear form
  $f$ on $V$ vanishing on $W$ is a composition
  \begin{displaymath}
    V\to V/W\xrightarrow{\ \bar{f}}K
  \end{displaymath}
  where $\bar{f}$ is continuous by Theorem \ref{thm:2.2.2}. Hence $f$
  is also continuous.
\end{prf}
Let $N$ be an open and balanced neighborhood of $0$ in the
topological vector space $V$. Then there is a unique function $p:V\to
\mathbf{R}_{+}$ (the non-negative reals) such that
  \begin{equation}
    \label{eq:2.2.4} p(ax)=|a|p(x)\quad a\in K,x\in V;\qquad N=\left\{
x\in V;p(x)<1 \right\}.
  \end{equation} Moreover, the function $p$ is continuous at $0$. In
fact, if $N$ is defined by $p$ in this way and if $t>0$, then
  \begin{displaymath} p(x)=tp(x/t)\geq t \text{ if } x/t\notin
N;\qquad p(x)<t\text{ if }x/t\in N,
  \end{displaymath} hence
  \begin{equation}
    \label{eq:2.2.5} p(x)=\inf \left\{ t;x/t\in N \right\}.
  \end{equation} Conversely, the function defined by (\ref{eq:2.2.5})
obviously satisfies (\ref{eq:2.2.4}) if $N$ is balanced and
absorbing. If $p(x)<1$ we have $x/t\in N$ for some $t<1$, hence $x\in
N$, and if $p(x)\geq1$ then $x/t\notin N$ if $t<1$. Since $N$ is open
we may conclude that $x\notin N$. Finally $p(x)<\varepsilon$ if
$x\in\varepsilon N$, so $p$ is continuous at $0$.

If $N_1$ is another neighborhood of $0$ of the same kind, and if
$N_1+N_1\subset N$ as in (\ref{eq:2.2.2}),we have $p(x+y)<1$ if
$p_1(x)<1$ and $p_1(y)<1$, where $p_1$ is defined by
(\ref{eq:2.2.5})with $N$ replaced by $N_1$. In view of the homogeneity
it follows that
\begin{displaymath}
  p(x+y)\leq \max\,(p_1(x),\,p_1(y)).
\end{displaymath}
Conversely, if we give a family $\mathcal{P}$ of function from $V$ to
$\mathbf{R}_+$ satisfying (\ref{eq:2.2.4}) such that for every $p\in\mathcal{P}$
\begin{displaymath}
  p(x+y)\leq C\max(p_1(x),p_1(y)),\quad x,y\in V,
\end{displaymath}
for some $p_1\in\mathcal{P}$ and some constant $C$, then the finite
intersections of the sets
\begin{displaymath}
  N_{p,\epsilon}=\left\{ x;p(x)<\epsilon \right\},\qquad\epsilon>0,\quad p\in\mathcal{P},
\end{displaymath}
are a basis of neighborhood of $0$ for a vector space topology in
$V$. The verification is left for the reader.
\begin{exa}
  Let $V$ be the space of $K$ valued continuous functions on
  $[0,1]\subset \mathbf{R}$,and set for some $r>0$
  \begin{displaymath}
    \|f\|_r=\left( \int|f|^r\,dx \right)^{1/r}.
  \end{displaymath}
Then $\|f\|_r$ satisfies (\ref{eq:2.2.4}) and
\begin{displaymath}
  \|f+g\|_r\leq 2^{(r+1)/r}\max(\|f\|_r,\|g\|_r).
\end{displaymath}
Thus the sets $\left\{ f;\|f\|_r<\epsilon \right\}$ form a basis for
neighborhood of $0$ in a vector space topology.
\end{exa}
The general spaces discssed so far are of little use in analysis so we
shall narrow down our discussion to more special classes of spaces
which occur frequently and have useful properties. The first condition
we impose is the existence of a fundamental system of convex
neighborhoods. This is required for the application of Hahn-Banach
theorem proved in Section \ref{sec:1.4}. 
\newcommand{\tvs}{Topological vector spaces}
\newcommand{\lctvs}{Locally convex topological vector spaces}
\newcommand{\sns}{Semi-normed spaces} \newcommand{\ns}{Normed spaces}
\newcommand{\phs}{Pre-Hilbert spaces} \newcommand{\fs}{Fr\'echet
  spaces} \newcommand{\Bs}{Banach spaces} \newcommand{\hs}{Hilbert
  spaces} \newcommand{\comb}{
\begin{tabular}{c}
  \tvs\\$\Downarrow$\\\lctvs
\end{tabular}
}
\begin{center}
  \begin{tabular}{ccccc}
    &\multicolumn{4}{l}{\ \comb}\\
    $\Downarrow$ && \multicolumn{3}{c}{$\Downarrow$}\\
    &&\multicolumn{3}{c}{\sns}\\
    $\Downarrow$ && $\Downarrow$ && $\Downarrow$ \\
    &&\ns && \phs \\
    $\Downarrow$ && $\Downarrow$ && $\Downarrow$ \\
    \fs & $\Longrightarrow$ &\multicolumn{1}{l}{\Bs} &
    $\qquad\Longrightarrow \qquad$ &\multicolumn{1}{l}{\hs}
  \end{tabular}
\end{center}

%%% Local Variables: 
%%% mode: latex
%%% TeX-master: "alfa"
%%% End: 
