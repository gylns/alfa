\section{The index of linear transformations}
We shall now resume the discussion of the quantity which occurs in the
left-hand side of \eqref{eq:1.2.3}, without assuming that $\dim V_1$
and $\dim V_2$ are finite. If $T:V_1\to V_2$ is a linear map such that
either $\dim\Ker T$ or $\dim\coker T$ is finite, we thus
define the index of $T$ by
\begin{equation}
  \label{eq:1.3.1}
  \ind T=\dim\Ker T-\dim\coker T.
\end{equation}
The index has a stability property generalizing \eqref{eq:1.2.3}.
\begin{thm}
  \label{thm:1.3.1}
  If $T$ and $S$ are linear maps from $V_1$ and $V_2$ such that
  $\ind T$ is defined and $S$ has finite rank, then
  $\ind (T+S)$ is defined and
  \begin{equation}
    \label{eq:1.3.2}
    \ind T=\ind (T+S).
  \end{equation}
\end{thm}
Before the proof of Theorem \ref{thm:1.3.1} we give another property
of the indexer, sometimes referred to as the logarithmic law, which is
convenient in computations involving the index.
\begin{thm}
  \label{thm:1.3.2}
  Let $T_1:V_1\to V_2$ and $T_2:V_2\to V_3$ be linear maps. For the
  linear map $T_2T_1:V_1\to V_3$ we have
  \begin{equation}
    \label{eq:1.3.3}
    \ind (T_2T_1)=\ind T_2+\ind T_1
  \end{equation}
provided that the right-hand side is defined, that is, either
$\dim\Ker T_j<\infty$ for $j=1,2$ or $\dim\coker T_j<\infty$
for $j=1,2$.
\end{thm}
\begin{prf}
  It suffices to verify the exactness of the complex
  \begin{equation}
    \label{eq:1.3.4}
    \begin{aligned}
      0\to\Ker T_1\xrightarrow{\ i}\Ker T_2T_1&\xrightarrow{\ T'_1}\Ker
      T_2\xrightarrow{\ q}V_2/\im T_1
      \\
      &\xrightarrow{\ T'_2}V_3/\im T_2T_1\xrightarrow{\ q}V_3/\im T_2\to0,
    \end{aligned}
  \end{equation}
where $i$ and $q$ denote inclusion and quotient maps and $T'_1$ and
$T'_2$ are derived from $T_1$ and $T_2$ in an obvious way. For then it
follows that the left-hand side of \eqref{eq:1.3.3} is defined if the
\eqref{eq:1.3.4} is an exact complex is left for the reader except for
the exactness at $V_2/\im T_1$ which may be somewhat less trivial than
the other statements. So assume that $x_2\in V_2$ and that the class
of $x_2$ in $V_2/\im T_1$ is mapped to $0$ by $T'_2$. This means that
$T_2x_2=T_2T_1x_1$ for some $x_1\in V_1$. Thus $x_2-T_1x_1\in\Ker
T_2$, and since $x_2-T_1x_1$ is equal to $x_2$ modulo $\im T_1$, this
proves the statement. 
\end{prf}
The following special case of Theorem \ref{thm:1.3.2} is often useful:
\begin{cor}
  \label{thm:1.3.3}
  Let $T:V_1\to V_2$ be a linear map such that $\ind T$ is
  defined, let $W_1$ be a subspace of $V_1$ of finite codimension and
  $W_2$ of finite dimension. Let $i:W_1\to V_1$ and $q:V_2\to V_2/W_2$
  be the inclusion and quotient maps. Then the index of $qTi:W_1\to
  V_2/W_2$ is defined and equals
  \begin{displaymath}
    \ind T+\dim W_2-\codim W_1.
  \end{displaymath}
\end{cor}
\begin{proof}[Proof of Theorem \ref{thm:1.3.1}:]
  By hypothesis $W_1=\Ker S$ has finite codimension. With the notation
  of Corollary \ref{thm:1.3.3} we have
  $\ind (Ti)=\ind T+\ind i$, if $i$ is the
  inclusion $W_1\to V_1$.  Since $Si=0$, it follows that
  \begin{displaymath}
    \ind ((T+S)i)=\ind (Ti)=\ind T+\ind i,
  \end{displaymath}
  which shows that either $\Ker(T+S)$ or $\coker (T+S)$ has
  finite dimension, for by \eqref{eq:1.2.9} we have
  $\dim\Ker(T+S)\leq\dim\Ker((T+S)i)+\codim W_1$, and
  $\dim\coker (T+S)\leq\dim\coker ((T+S)i)$. Thus
  $\ind (T+S)$ is defined, and another application of Theorem
  \ref{thm:1.3.2} shows that
  \begin{displaymath}
   \ind ((T+S)i)=\ind (T+S)+\ind i,
  \end{displaymath}
  hence that $\ind (T+S)=\ind T$.
\end{proof}
When $T$ is a linear map $V\to V$ and $\dim V<\infty$ we know that
$\ind T=0$. This is not always true in the infinite
dimensional case, however, which is one of the reasons for the
interest of the index.
\begin{emp}
  Let $V=K^{\{1,2,\dots\}}$, that is, the set of all sequences
    $x=(x_1,x_2,\dots)$ with elements in $K$, the vector operations
    being defined coordinatewise. Let $n$ be a fixed integer and set
    \begin{displaymath}
      Tx=(x_{n+1},x_{n+2},\dots)
    \end{displaymath}
where coordinates with index $\leq0$ should be read as $0$. Then we
have
\begin{displaymath}
  \dim\Ker T=\max(n,0),\qquad\dim\coker T=\max(-n,0),
\end{displaymath}
and it follows that $\ind T=n$.
\end{emp}

%%% Local Variables: 
%%% mode: LaTeX
%%% TeX-master: "alfa"
%%% End: 
