\section{Topological and metric spaces}
Let us first recall that a set $E$ is called a topological space if
there is given a family $\O$ of subsets, which are said to be
\emph{open}, such that 
\begin{enumerate}[label=\alph*),itemindent=!,leftmargin=0pt]
\item\label{item:2.1} $E\in\O$,
\item\label{item:2.2} the union of the sets in any subfamily of $\O$ is always a set in $\O$,
\item\label{item:2.3} the intersection of finitely many sets in $\O$ is also in $\O$.
\end{enumerate}
The space $E$ is called a Hausdorff space if in addition
\begin{enumerate}[resume*]
\item\label{item:2.4} for two arbitrary different points $x,y\in E$ one can always
  find two disjoint sets in $\O$ containing $x$ and $y$ respectively.
\end{enumerate}\par
The complements in $E$ of the open sets are said to be
\emph{closed}. According to \ref{item:2.2} and \ref{item:2.3}
\emph{arbitrary intersections and finite unions of closed sets are
  closed.} For every set $M\subset E$ the intersection $\overline{M}$
of all closed sets containing $M$ is therefore closed, hence the
smallest closed set containing $M$.

A subset $N$ of $E$ is called a neighborhood of $x\in E$ (and $x$ is
said to be an interior point of $N$) if $x\in O\subset N$ for some
$O\in\O$. The neighborhoods of $x$ have the following properties:
\begin{enumerate}[label=\roman*),itemindent=!]
\item\label{item:2.5} $E$ is a neighborhood of $x$, and $x$ belongs to
  every neighborhood of $x$,
\item\label{item:2.6} the intersection of a finite number of
  neighborhoods of $x$ is a neighborhood of $x$,
\item\label{item:2.7} if $N$ is a neighborhood of $x$ and $N'\supset
  N$, then $N'$ is a neighborhood of $x$,
\item\label{item:2.8} if $N$ is a neighborhood of $x$ then there is
  another neighborhood $N'$ of $x$ such that $N$ is a neighborhood of
  every $y\in N'$.
\end{enumerate}
From the neighborhoods one can reconstruct the open sets: If $O$ is a
subset of $E$ then $O\in\O$ if (and only if) $O$ is a neighborhood of
all of its points. In fact, for every $x\in O$ we can then find
$O_x\in\O$ with $x\in O_x\subset O$. Hence $\cup_{x\in O}O_x=O$ is in
$\O$ by condition \ref{item:2.2}.

Suppose now instead that for every $x\in E$ we are given a family
$\V_x$ of subset of $E$ called neighborhoods of $x$, so that
\ref{item:2.5} and \ref{item:2.6} are fulfilled. We can then define a
topology $\T$ in $E$ by declaring that a set $O\subset E$ is open if
for every $x\in O$ we have $N\subset O$ for some $N\in\V_x$. Then
\ref{item:2.1} follows from \ref{item:2.5}, \ref{item:2.2} is trivial,
and \ref{item:2.3} follows from \ref{item:2.6} or even the weaker
version
\begin{enumerate}[label=\roman*)$'$,start=2]
\item\label{item:2.9} the intersection of a finite number of
  neighborhoods of $x$ contains a neighborhood of $x$.
\end{enumerate}
If $\N$ is a neighborhood of $x$ in the topology $\T$ thus defined,
then we can find an open set $O$ with $x\in O\subset\N$, hence some
$N'\in\V_x$ with $x\in N'\subset O\subset\N$, so $\N$ contains one of
our original ``neighborhoods''. Now suppose that \ref{item:2.8} is
fulfilled, and let $N\in\V_x$. Denote by $O$ the set of all $y\in N$
such that $N\in\V_y$. By \ref{item:2.8} we can then find $N_y'\in\V_y$
such that $N\in\V_z$ for every $z\in N'_y$, thus $N'_y\subset O$. This
means that $O$ is open in the topology $\T$ and since $x\in O\subset
N$ we see that $N$ is also a neighborhood of $x$ in the topology
$\T$. The neighborhoods of $x$ in the topology $\T$ are therefore
precisely the sets containing some set belonging to $\V_x$. If
\ref{item:2.7} is fulfilled the neighborhoods are precisely those we
were given from the beginning.

To define a topology we can therefore either give a family of open
sets satisfying \ref{item:2.1},~\ref{item:2.2},~\ref{item:2.3} or give
the families $\V_x$ of neighborhoods satisfying at least
\ref{item:2.5},~\ref{item:2.9},~\ref{item:2.8}. If condition
\ref{item:2.7} is not required one calls the sets $\V_x$ a
\emph{neighborhood basis} or a \emph{fundamental system of
  neighborhoods}. The neighborhoods of $x$ in the topology defined are
precisely the sets which contain some set belonging to $\V_x$.

Any subset $E_1$ of $E$ is itself a topological space with the
restriction of the topology in $E:$ The open sets in $E_1$ are by
definition the sets $O\cap E_1$ with $O\in\O$.

A map $f:E_1\to E_2$ between topological spaces is called
\emph{continuous} if for every open set $O\subset E_2$ the
\emph{inverse image}
\begin{displaymath}
  f^{-1}(O)\eqdef\{x\in E_1; f(x)\in O\}
\end{displaymath}
is an open set. Equivalently this means that for every neighborhood
$N_2$ of $f(x)$ one can find a neighborhood $N_1$ of $x$ with
$f(N_1)\subset N_2$.

A sequence $x_j\in E$ is said to converge to $x\in E$ if for any
neighborhood $N$ of $x$ all but a finite number of the points $x_j$
belong to $N$. If the Hausdorff separation axiom \ref{item:2.4} above
is satisfied, then this cannot happen for more than one value of
$x$. A closed set $F\subset E$ contains all limits of sequences
contained in $F$, for $E\setminus F$ is a neighborhood of all of it
points and contains no element of such a sequence.

A topology can sometimes bedefined by means of a \emph{metric}, that
is, a function of $d$ on $E\times E$ with values in the non-negative
reals such that
\begin{enumerate}[label=\alph*),itemindent=!,leftmargin=0pt]
\item\label{item:2.10} $d(x,z)\leq d(x,y)+d(y,z)$, (the triangle
  inequality),
\item\label{item:2.11} $d(x,y)=d(y,x)$,
\item\label{item:2.12} $d(x,y)=0\Longleftrightarrow x=y$,
\end{enumerate}
for all $x,y,z\in E$. The metric gives rise to a system of
neighborhoods of $x$,
\begin{displaymath}
  N_{x,\epsilon} = \{y\in E; d(x,y)<\epsilon\},\qquad
  0<\epsilon\leq\infty,\ x\in E,
\end{displaymath}
satisfying \ref{item:2.5},~\ref{item:2.6},~\ref{item:2.8} above, so
the metric defines a topology. It follows from \ref{item:2.10} that
$d(x,y)$ is a continuous function of $x$ (or $y$) in this topology. It
is clear that the topology satisfies the Hausdorff separation
condition \ref{item:2.4} above. (Note that all topologies cannot be
defined by a metric, and that different metrics may define the same
topology.)

A metric space is called \emph{complete} if for every \emph{Cauchy
  sequence}, that is, every sequence $x_n\in E$ with $d(x_n,x_m)\to 0$
as $n,m\to\infty$, there is an element $x\in E$ such that $x_n\to x$,
that is, $d(x_n,x)\to 0$. The real numbers $\mathbf{R}$ and the
complex numbers $\mathbf{C}$ with the usual distance $d(x,y)=|x-y|$
for $x,y\in \mathbf{C}$ are examples of complete metric spaces.

Some of the most important theorems in functional analysis follow from
a fundamental classical theorem of Baire concerning complete metric
spaces:
\begin{thm}\label{thm:2.1.1}
  (Baire) Let $E$ be a complete metric space, and let $F_n,\ 
  n=1,2,\dots,$ be closed subsets containing no interior points. Then
  the union $\cup_1^{\infty}F_n$ has no interior point either.
\end{thm}
\begin{prf}
  For any $x\in E$ and $\epsilon>0$ we want to show that
  \begin{displaymath}
    N_{x,\epsilon} = \left\{ y\in E;d(x,y)\leq\epsilon \right\}
  \end{displaymath}
is not contained in $\cup F_n$. To do so we want to choose sequences
$x_j\in E$ and $\epsilon_j>0$ so that
\begin{displaymath}
  N_{x,\epsilon}\supset N_{x_1,\epsilon_1}\supset
  N_{x_2,\epsilon_2}\supset\cdots,\qquad F_j\cap
  N_{x_j,\epsilon_j}=\emptyset \quad \text{for every } j.
\end{displaymath}
When $j<k$ this implies that $x_k\in N_{x_j,\epsilon_j}$, hence
$d(x_j,x_k)<\epsilon_j$, so $x_k$ is a Cauchy sequence if
$\epsilon_j\to 0$. The limit $y$ belongs to $N_{x_j,\epsilon_j}$ for
every $j$ since $N_{x_j,\epsilon_j}$ is closed. Hence $y\notin\cup
F_j$ so the theorem will follow if we show that the sequences
$x_j,\epsilon_j$ can be constructed. Suppose that
$x_1,\epsilon_1,\dots,x_{j-1},\epsilon_{j-1}$ have already been
chosen. By hypothesis the ball $N_{x_{j-1},\epsilon_{j-1}/2}$ contains
some point $x_j$ which is not in $F_j$. Since $F_j$ is closed we can
choose $\epsilon_j$ so that $F_j\cap N_{x_j,\epsilon_j}=\emptyset$. If
$\epsilon_j<\epsilon_{j-1}/2$, we have $N_{x_j,\epsilon_j}\subset
N_{x_{j-1},\epsilon_{j-1}}$ by the triangle inequality, and
$\epsilon_j<2^{1-j}\epsilon_1\to 0$. This proves the theorem.
\end{prf}
\begin{exa}
  We give an example with $E=\mathbf{R}$. Let $f$ be a function from
  $\mathbf{R}$ to $\mathbf{R}$ which is differentiable at every
  point. The example $f(x)=x^2\sin(x^{-2}),\ x\neq0;\ f(0)=0$; shows
  that $\varlimsup_{y\to x}|f'(y)|$ need not be finite for every
  $x$. However, we shall prove that
  \begin{displaymath}
    E=\left\{ x\in\mathbf{R};\varlimsup_{y\to x}|f'(y)|<\infty \right\}
  \end{displaymath}
is an open dense set (that is, the complement is closed and has no
interior point). That $E$ is open is obvious. Let $I$ be a closed
interval in $\mathbf{R}$, not reduced to a point, and set
\begin{displaymath}
  F_n=\left\{ x\in I;|f(x)-f(y)|\leq n|x-y|,y\in I \right\}\subset
  \left\{ x\in I;|f'(x)|\leq n \right\}.
\end{displaymath}
Since $f$ is continuous it is clear that $F_n$ is closed, and $\cup
F_n=I$ since $f'(x)$ exists for every $x$. Hence $F_n$ has an interior
point for some $n$, so $I\cap E$ is not empty, which means that $E$ is
dense in $\mathbf{R}$.
\end{exa}
\begin{dfn}
  A subset $A$ of complete metric space is said to be of the
  \emph{first category} if there exist countably many closed sets
  $F_1,F_2,\dots$ in $E$ without interior points such that
  $A\subset\cup_1^{\infty}F_j$. All other sets are said to be of the
  \emph{second category}.
\end{dfn}
The definition implies that any countable union of sets of the first
category is of the first category, and so is any subset of a set of the
first category. By Theorem \ref{thm:2.1.1} a set of the first category
has no interior point, so the complement is dense. One is therefore
justified in thinking of sets of the first category as quite small
although they may of course be dense (such as the set of rational
numbers $\subset\mathbf{R}$).

We shall finally recall the main facts concerning compact spaces. A
topological space $E$ is called \emph{compact} if it is a Hausdorff
space and the Borel-Lebesgue lemma is valid, that is, if for every
family of open subsets $O_{\alpha},\alpha\in A$, with
$\cup_AO_{\alpha}=E$ it is possible to find a finite subfamily
$O_{\alpha_1},\dots,O_{\alpha_n}$ with union equal to $E$. Here $A$
may be any set of indices. An equivalent property is that for closed
subsets $F_{\alpha}$ of $E$ with $\cap_AF_{\alpha}=\emptyset$ a finite
number of the sets already have an empty intersection. This follows by
considering the complements. Negation gives a third equivalent and
useful statement:

If $F_{\alpha}$ are closed subsets of the compact set $E$ and if any
finite number of the sets $F_{\alpha}$ have a non-empty intersection,
then all have a non-empty intersection.

It is obvious that \emph{a closed subset of a compact space is
  compact}. The converse is contained in the following:
\begin{thm}\label{thm:2.1.3}
  (i) Every compact subset of a Hausdorff space is closed. (ii) If
  $f:E\to F$ is continuous, $E$ is compact and $F$ is Hausdorff, then
  $F(E)$ is compact. (iii) If in addition $f$ is injective, then $f$
  is a homeomorphism $E\to f(E)$, that is, the inverse is also
  continuous. (iv) If $E$ is compact, then every $x\in E$ has a
  fundamental system of compact neighborhoods.
\end{thm}
\begin{prf}
  (i) Let $K$ be a compact subset of the Hausdorff space $E$ and let
  $x\in E\setminus K$. For every $y\in K$ we can choose disjoint open
  sets $O'_y\ni x$ and $O''_y\ni y$. Since $K$ is compact, we can find
  $y_1,\dots,y_k$ so that $K\subset O''_{y_1}\cup\cdots\cup O''_{y_k}
  $. Then $O'_{y_1}\cap\cdots\cap O'_{y_k}$ is an open set containing
  $x$ which does not intersect $K$. Hence $E\setminus K$ is an open
  set. (ii) If $O_{\alpha}$ are open subsets of $F$ with
  $f(E)\subset\cup O_{\alpha}$, then $E\subset\cup
  f^{-1}(O_{\alpha})$, so $E\subset\cup_1^kf^{-1}(O_{\alpha_j})$ for
  suitable $\alpha_1,\dots,\alpha_k$. This implies that
  $f(E)\subset\cup O_{\alpha_j}$ so $f(E)$ is compact set since it is
  clearly Hausdorff. (iii) From (ii) it follows that $f$ maps closed
  subsets of $E$ to closed subsets of $f(E)$, hence open subsets to
  open subsets, which means that $f^{-1}$ is continuous. (iv) If $O$
  is an open neighborhood of $x$ then the proof of (i) applied to
  $K=E\setminus O$ shows that there are disjoint open sets $O'\ni x$
  and $O''\supset K$; then $O'\subset N=E\setminus O''\subset O$, and
  since $N$ is closed, hence compact, the statement is proved.
\end{prf}

Let $A$ be an arbitrary set and let $E_{\alpha}$ be a compact set for
every $\alpha\in A$. The infinite product
\begin{displaymath}
  E=\prod_{\alpha\in A}E_{\alpha}
\end{displaymath}
is defined as the set of all systems $\left\{ x_{\alpha}
\right\}_{\alpha\in A}$ with $x_{\alpha}\in E_{\alpha}$. Let
$p_{\alpha}$ be the projection $E\to E_{\alpha}$ on the $\alpha$th
component. A topology in $E$ is defined by taking as a basis for open
sets the finite intersections of sets of the form
$p_{\alpha}^{-1}O_{\alpha}$ where $O_{\alpha}$ is open in
$E_{\alpha}$; note that this makes the projections $p_{\alpha}$
continuous.
\begin{thm}
  (Tychonov) The inifinite product $E=\prod_{\alpha\in A}E_{\alpha}$
  is compact if each $E_{\alpha}$ is compact.
\end{thm}
\begin{prf}
  Let $\F$ be a family of closed subsets of $E$ such that finitely
  many members of $\F$ never have an empty intersection. Using Zorn's
  lemma we can extend $\F$ to a maximal family $\F'$ having the same
  property. In particular, finite intersections of sets in $\F'$ are
  themselves in $\F'$. Since $E_{\alpha}$ is compact we can find
  $x_{\alpha}\in E_{\alpha}$ so that
  $x_{\alpha}\in\overline{p_{\alpha}F}$ for every $F\in\F'$, where the
  bar denotes closure. If $U_{\alpha}$ is a compact neighborhood of
  $x_{\alpha}$, then it follows that $p_{\alpha}^{-1}U_{\alpha}\in\F'$
  since $\F'$ is maximal. Hence $x_{\alpha}$ is uniquely
  determined. Choosing all $U_{\alpha}$ except finitely many equal to
  $E_{\alpha}$, we obtain
  \begin{displaymath}
    \prod U_{\alpha}=\bigcap
    p_{\alpha}^{-1}U_{\alpha}\in\F',\quad\text{hence } \left( \prod
      U_{\alpha} \right)\cap F\neq\emptyset \quad \text{if } F\in\F'
  \end{displaymath}
  Thus an arbitrary neighborhood of $x=\left\{ x_{\alpha}
  \right\}_{\alpha\in A}$ intersects $F$, for every $F\in\F'$, and
  since $F$ is closed this implies that $x\in F$. The proof is complete.
\end{prf}

%%% Local Variables: 
%%% mode: latex
%%% TeX-master: "alfa"
%%% End: 
